\documentclass{article}
\usepackage{listings}
\usepackage{xcolor}
\usepackage[margin=1in]{geometry}
\usepackage{hyperref}

\definecolor{codegreen}{rgb}{0,0.6,0}
\definecolor{codegray}{rgb}{0.5,0.5,0.5}
\definecolor{codepurple}{rgb}{0.58,0,0.82}
\definecolor{backcolour}{rgb}{0.95,0.95,0.92}

\lstdefinestyle{cppstyle}{
    backgroundcolor=\color{backcolour},   
    commentstyle=\color{codegreen},
    keywordstyle=\color{blue},
    numberstyle=\tiny\color{codegray},
    stringstyle=\color{codepurple},
    basicstyle=\ttfamily\small,
    breakatwhitespace=false,         
    breaklines=true,                 
    captionpos=b,                    
    keepspaces=true,                 
    numbers=left,                    
    numbersep=5pt,                  
    showspaces=false,                
    showstringspaces=false,
    showtabs=false,                  
    tabsize=2,
    language=C++
}

\lstset{style=cppstyle}

\title{C++ Programming Guide\\Using Examples from a Hexagon Game}
\author{Game Development Team}
\date{\today}

\begin{document}

\maketitle
\tableofcontents

\section{Introduction to C++}

C++ is a powerful, general-purpose programming language that extends the C language with object-oriented features. It was designed by Bjarne Stroustrup and is widely used for system/software development, game programming, and performance-critical applications.

This guide will teach C++ concepts using real-world examples from our hexagon-based game project.

\section{Basic C++ Concepts}

\subsection{Classes and Objects}

Classes are the building blocks of object-oriented programming in C++. A class is a blueprint that defines attributes (data members) and behaviors (member functions) of objects.

\begin{lstlisting}[caption=Basic Class Definition (Character.h)]
class Character {
    public:
        Character(int q, int r);
        virtual ~Character() = default;
        
        // Position setting methods
        void setPosition(const sf::Vector2f& pixelPos);
        void setHexCoord(int q, int r);
        
    private:
        int mQ; // Q coordinate in hex grid
        int mR; // R coordinate in hex grid
        
        int health;
        int maxHealth;
        int attackPower;
};
\end{lstlisting}

The \texttt{Character} class defines the basic properties and behaviors of game characters. Objects are instances of classes.

\subsection{Encapsulation}

Encapsulation is hiding implementation details and restricting access to data through access specifiers:

\begin{itemize}
    \item \texttt{public}: Members accessible from anywhere
    \item \texttt{protected}: Members accessible within the class and derived classes
    \item \texttt{private}: Members accessible only within the class
\end{itemize}

\begin{lstlisting}[caption=Encapsulation Example (Building.h)]
class Building {
    public:
        // Public interface - accessible to all
        void draw(sf::RenderWindow& window) const;
        void setPosition(const sf::Vector2f& position);
        sf::Vector2f getPosition() const { return mPosition; }
    
    protected:
        // Protected members - accessible to derived classes
        void initializeShape();
        bool tryLoadTexture(const std::string& imagePath);
        static constexpr float STANDARD_SIZE = 25.0f;
        
    private:
        // Private members - only accessible within Building
        sf::RectangleShape mShape;
        sf::Vector2f mPosition;
        std::unique_ptr<sf::Texture> mTexture;
        bool mHasTexture;
};
\end{lstlisting}

\subsection{Headers and Source Files}

C++ programs typically split class declarations (headers) from implementations (source files):

\begin{lstlisting}[caption=Header File (Soldier.h)]
#ifndef SOLDIER_H
#define SOLDIER_H

#include "Character.h"

class Soldier final : public Character {
public:
    Soldier(int q, int r);
    
    // Methods declarations
    CharacterType getType() const override { return CharacterType::Soldier; }
    void attack(Character* target);
    void move(Hexagon* targetHex);

protected:
    void doRender(sf::RenderWindow& window) const override;
};

#endif
\end{lstlisting}

\begin{lstlisting}[caption=Source File (Soldier.cpp)]
#include "../../include/Soldier.h"
#include <iostream>

Soldier::Soldier(int q, int r) : Character(q, r) {
    loadTexture(SOLDIER_TEXTURE);
}

void Soldier::doRender(sf::RenderWindow& window) const {
    Character::doRender(window);
    // Soldier-specific rendering
}

void Soldier::attack(Character* target) {
    if (!target) return;
    std::cout << "Soldier attacks!" << std::endl;
}

void Soldier::move(Hexagon* targetHex) {
    if (!targetHex) return;
    setHexCoord(targetHex->getCoord());
    setPosition(targetHex->getPosition());
}
\end{lstlisting}

\section{Object-Oriented Programming}

\subsection{Inheritance}

Inheritance allows a class to reuse and extend the functionality of another class.

\begin{lstlisting}[caption=Inheritance Example]
// Base class
class Character {
    // Common functionality for all characters
};

// Derived class
class Soldier final : public Character {
    // Soldier-specific functionality
    // "final" means no further derivation allowed
};
\end{lstlisting}

\subsection{Polymorphism}

Polymorphism allows objects of different classes to be treated as objects of a common base class, with specific behaviors determined at runtime.

\begin{lstlisting}[caption=Polymorphism Example]
// Base class with virtual function
class Character {
public:
    virtual CharacterType getType() const = 0; // Pure virtual
    virtual void update(float deltaTime) {}
};

// Derived class implementations
class Soldier final : public Character {
public:
    CharacterType getType() const override { return CharacterType::Soldier; }
    void update(float deltaTime) override { /* Soldier update */ }
};

// Using polymorphism
void gameLoop(std::vector<Character*>& characters, float deltaTime) {
    for (auto character : characters) {
        character->update(deltaTime); // Calls appropriate version
    }
}
\end{lstlisting}

\subsection{Abstract Classes and Interfaces}

Abstract classes have at least one pure virtual function and cannot be instantiated.

\begin{lstlisting}[caption=Abstract Class Example]
class Character {
public:
    virtual CharacterType getType() const = 0; // Pure virtual
    virtual float getScaleFactor() const = 0; // Pure virtual
    
    // Concrete method using abstract methods
    void render(sf::RenderWindow& window) const {
        // Implementation that calls virtual methods
    }
};
\end{lstlisting}

\section{Modern C++ Features}

\subsection{Smart Pointers}

Smart pointers manage memory automatically to prevent leaks.

\begin{lstlisting}[caption=Smart Pointers Example (Game.h)]
// Use unique_ptr for owned Soldier
std::unique_ptr<Soldier> mSoldier;

// This is a non-owning reference, so keep as raw pointer
std::optional<Character*> mSelectedCharacter;

// Vector of unique_ptrs for owned cities
std::vector<std::unique_ptr<City>> mCities;
\end{lstlisting}

\subsection{Override and Final Specifiers}

\begin{lstlisting}[caption=Override and Final Example]
// Base class
class Character {
public:
    virtual void update(float deltaTime) {}
};

// Derived class - marked final, cannot be derived from further
class Soldier final : public Character {
public:
    // override ensures this overrides a base class method
    void update(float deltaTime) override { /* implementation */ }
};
\end{lstlisting}

\subsection{Non-Virtual Interface Pattern}

The NVI pattern provides a stable interface while allowing customization.

\begin{lstlisting}[caption=NVI Pattern Example]
class Building {
public:
    // Non-virtual public interface
    void draw(sf::RenderWindow& window) const {
        // Pre-conditions, common code...
        doDraw(window);
        // Post-conditions, common code...
    }
    
protected:
    // Virtual implementation method
    virtual void doDraw(sf::RenderWindow& window) const;
};
\end{lstlisting}

\section{Memory Management}

\subsection{RAII (Resource Acquisition Is Initialization)}

C++ uses RAII to tie resource management to object lifetime.

\begin{lstlisting}[caption=RAII Example]
// Resource (texture) is acquired in constructor
Building::Building(const sf::Vector2f& position) : 
    mPosition(position), 
    mHasTexture(false) {
    mTexture = std::make_unique<sf::Texture>();
    initializeShape();
}

// Resource is automatically released when object is destroyed
// No explicit cleanup code needed because of smart pointers
\end{lstlisting}

\subsection{Ownership Semantics}

Clear ownership semantics prevent memory leaks and double-free errors.

\begin{lstlisting}[caption=Ownership Example (Game.h)]
class Game {
private:
    // Game owns the soldier (unique_ptr)
    std::unique_ptr<Soldier> mSoldier;
    
    // Game doesn't own the selected character (raw pointer)
    std::optional<Character*> mSelectedCharacter;
    
    // Game owns all cities (vector of unique_ptrs)
    std::vector<std::unique_ptr<City>> mCities;
};
\end{lstlisting}

\section{Best Practices}

\subsection{Polymorphic Best Practices}

\begin{lstlisting}[caption=Polymorphic Best Practices]
// 1. Virtual destructors
virtual ~Character() = default;

// 2. Use override for all overridden methods
void update(float deltaTime) override;

// 3. Mark leaf classes as final
class Soldier final : public Character {...};

// 4. Non-Virtual Interface pattern
// Public non-virtual:
void render(sf::RenderWindow& window) const;
// Protected virtual:
virtual void doRender(sf::RenderWindow& window) const;

// 5. Pure virtual methods for required implementations
virtual BuildingType getType() const = 0;

// 6. Smart pointers for memory management
std::unique_ptr<sf::Texture> mTexture;

// 7. Clear ownership semantics
// Owner: std::unique_ptr<T>
// Non-owner: T* or std::optional<T*>
\end{lstlisting}

\section{Advanced C++ Concepts}

\subsection{Templates}

Templates enable generic programming, allowing algorithms and classes to work with any data type.

\begin{lstlisting}[caption=Templates Example (Standard Library Usage)]
// Using vector template with Building pointers
std::vector<std::unique_ptr<Building>> buildings;

// Using make_unique template function
auto soldier = std::make_unique<Soldier>(q, r);
\end{lstlisting}

\subsection{Const Correctness}

Functions that don't modify object state should be marked const.

\begin{lstlisting}[caption=Const Correctness Example]
// Const methods - don't modify object state
sf::Vector2f getPosition() const { return mPosition; }
void render(sf::RenderWindow& window) const;
CharacterType getType() const override;

// Non-const methods - modify object state
void setPosition(const sf::Vector2f& position);
void update(float deltaTime);
\end{lstlisting}

\section{Conclusion}

This guide covered fundamental and advanced C++ concepts with practical examples from a real game project. The hexagon game demonstrates many C++ best practices including class design, inheritance, polymorphism, memory management, and modern C++ features.

\end{document}
